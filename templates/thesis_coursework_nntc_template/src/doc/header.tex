%%% Здесь выбираются необходимые графы
\documentclass[russian,utf8,pointsection,nocolumnsxix,nocolumnxxxi,nocolumnxxxii]{eskdtext}
\usepackage{fontspec}
\defaultfontfeatures{Mapping=tex-text} % Для того чтобы работали стандартные сочетания символов ---, --, << >> и т.п.

%%% Что бы работал eskdx и некоторые другие пакеты LaTeX
\usepackage{xecyr}

%%% Для работы шрифтов
\usepackage{xunicode,xltxtra}

\usepackage{xcolor} % для управления цветами
\usepackage{tcolorbox} % для создания рамок
\usepackage{tikz} % Подключение пакета tikz

\usepackage{hyperref}

%%% Для работы с русскими текстами (расстановки переносов, последовательность комманд строго обязательна)
%\usepackage{polyglossia}
%\setdefaultlanguage{russian}
%\newfontfamily{\cyrillicfontt}{GOST_B}
%\set{GOST_type_A}
\setmainfont{Times New Roman}
\setromanfont{Times New Roman}
\setsansfont{Arial}
\setmonofont{Courier New}

\usepackage[numbercenter]{eskdplain} %отключить рамки

% polyglossia only
% \newfontfamily\cyrillicfont{GOST_type_A} 
% \newfontfamily\cyrillicfontrm{GOST_type_A}
% \newfontfamily\cyrillicfonttt{GOST_type_A}
% \newfontfamily\cyrillicfontsf{GOST_type_A}
%\defaultfontfeatures{Mapping=tex-text}

%%% Для работы со сложными формулами
\usepackage{amsmath}
\usepackage{amssymb}

%%% Что бы использовать символ градуса (°) - \degree
\usepackage{gensymb}

% Настройка заголовков
\ESKDsectStyle{section}{\sffamily\Large\bfseries} % Разделы — Arial, крупный, жирный
\ESKDsectStyle{subsection}{\sffamily\large\bfseries} % Подразделы — Arial, меньше, жирный
\ESKDsectStyle{subsubsection}{\sffamily\normalsize} % Подподразделы — Arial, нормальный размер

% Переопределение шрифта заголовков
\renewcommand{\ESKDtitleFontI}{\rmfamily\normalsize} % Без засечек, жирный, крупный
\renewcommand{\ESKDtitleFontII}{\rmfamily\normalsize} % не жирный
\renewcommand{\ESKDtitleFontIII}{\rmfamily\normalsize} % не жирный
\renewcommand{\ESKDtitleFontIV}{\rmfamily\normalsize} % не жирный
\renewcommand{\ESKDtitleFontV}{\rmfamily\normalsize} % не жирный
\renewcommand{\ESKDtitleFontVI}{\rmfamily\normalsize} % не жирный
\renewcommand{\ESKDtitleFontVII}{\rmfamily\normalsize} % не жирный
\renewcommand{\ESKDtitleFontVIII}{\rmfamily\normalsize} % не жирный
\renewcommand{\ESKDtitleFontIX}{\rmfamily\normalsize} % не жирный
\renewcommand{\ESKDtitleFontX}{\rmfamily\normalsize} % не жирный

\makeatletter
\newcommand{\ESKDNNTCspeciality}[1]{%
  \def\ESKDNNTC@speciality{#1}%
}
\newcommand{\ESKDNNTCqualification}[1]{%
  \def\ESKDNNTC@qualification{#1}%
}
\newcommand{\ESKDNNTCsubject}[1]{%
  \def\ESKDNNTC@subject{#1}%
}

\renewcommand{\ESKDtheTitleFieldIV}{%
  \MakeUppercase{\ESKDtheTitle}\par
  \vspace{2mm}
  \ESKDNNTC@subject\par
  \vspace{2mm}
  \ESKDtheDocName
}

\renewcommand{\ESKDtheTitleFieldI}{%
  \ESKDtheDepartment\par
  \ESKDtheCompany\par
  \ESKDNNTC@speciality\par
  \ESKDNNTC@qualification
}
\makeatother

\newcommand{\ESKDNNTCtitle}[3]{%
  \renewcommand{\ESKDtheTitleFieldVIII}{%
    \noindent
    \begin{minipage}[t]{0.45\textwidth}
      Выполнил студент группы\\
      #1\\
      #2\\[2mm]
    \end{minipage}%
    \hfill
    \begin{minipage}[t]{0.45\textwidth}
      Проверил преподаватель\\
      #3\\
      Проект защищен с оценкой: \makebox[6mm]{\hrulefill}\\
      Дата защиты: \underline{\hspace{30mm}}\\
    \end{minipage}%

    \vspace{5mm}

    \noindent
    \begin{minipage}[t]{0.45\textwidth}
      \mbox{Подпись студента:} \underline{\hspace{20mm}}
    \end{minipage}%
    \hfill
    \begin{minipage}[t]{0.45\textwidth}
      \mbox{Подпись преподавателя:} \underline{\hspace{20mm}}
    \end{minipage}%
  }
}

\newcommand{\ESKDNNTCcitydate}[2]{%
  \def\@ESKDNNTCcity{#1}
  \ESKDdate{#2}
}

\renewcommand{\ESKDtheTitleFieldX}{\@ESKDNNTCcity, \ESKDtheYear}


%%% Для переноса составных слов
%\XeTeXinterchartokenstate=1
\XeTeXcharclass `\- 24
\XeTeXinterchartoks 24 0 ={\hskip\z@skip}
\XeTeXinterchartoks 0 24 ={\nobreak}

%%% Ставим подпись к рисункам. Вместо «рис. 1» будет «Изображение 1»
\addto{\captionsrussian}{\renewcommand{\figurename}{Изображение}}
%%% Убираем точки после цифр в заголовках
\def\russian@capsformat{%
  \def\postchapter{\@aftersepkern}%
  \def\postsection{\@aftersepkern}%
  \def\postsubsection{\@aftersepkern}%
  \def\postsubsubsection{\@aftersepkern}%
  \def\postparagraph{\@aftersepkern}%
  \def\postsubparagraph{\@aftersepkern}%
}



% Автоматически переносить на след. строку слова которые не убираются
% в строке
\sloppy

%%% Для вставки рисунков
\usepackage{graphicx}

%%% Для вставки интернет ссылок, полезно в библиографии
\usepackage{url}

%%% Подподразделы(\subsubsection) не выводим в содержании
\setcounter{tocdepth}{2}

%%% Каждый раздел (\section) с новой страницы
\let\stdsection\section
\renewcommand\section{\newpage\stdsection}

%%% В введении нумерация подразделов идёт с буквой «В» (например В.1)
\makeatletter
\renewcommand\thesubsection{\ifnum\c@section=0{В.\arabic{subsection}}\else{\arabic{section}.\arabic{subsection}}\fi}
\makeatother
