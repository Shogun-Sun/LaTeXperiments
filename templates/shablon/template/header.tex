%%% Здесь выбираются необходимые графы% Render all \texttt{...} as «...»
\let\oldtexttt\texttt
\renewcommand{\texttt}[1]{«\oldtexttt{#1}»}

\documentclass[russian,utf8,pointsection,reduceheight=2mm]{eskdtext}
\usepackage{fontspec}
\defaultfontfeatures{Mapping=tex-text} % Для того чтобы работали стандартные сочетания символов ---, --, << >> и т.п.

%%% Установка выравнивания текста по ширине (согласно ГОСТ ЕСКД)
\raggedbottom
% \flushbottom % Закомментировано для предотвращения растягивания контента по вертикали

\setlength{\parindent}{1.25cm}
\setlength{\parskip}{0pt}
\setlength{\leftskip}{0.5cm}
\setlength{\rightskip}{0.5cm}


%%% Что бы работал eskdx и некоторые другие пакеты LaTeX
\usepackage{xecyr}

%%% Для работы шрифтов
\usepackage{xunicode,xltxtra}

%%% Для отображения кода
\usepackage{listings}
\usepackage{xcolor}
\usepackage{enumitem}
\usepackage{placeins}
% Удаление вертикальных отступов до и после таблиц
\usepackage{etoolbox}
% \AtBeginEnvironment{table}{\vspace{-1em}} % Commented out or removed
\AtBeginEnvironment{table}{\renewcommand{\baselinestretch}{1}\selectfont}

% Настройка стиля листингов
\lstset{
  % basicstyle=\small\fontspec{3270 Nerd Font},
  basicstyle=\fontsize{10}{12}\fontspec{Liberation Mono},
  % numbers=left, % Removed line numbers
  % numberstyle=\tiny, % Removed line numbers style
  stepnumber=1,
  numbersep=5pt,
  backgroundcolor=\color{white},
  showspaces=false,
  showstringspaces=false,
  showtabs=false,
  frame=single,
  rulecolor=\color{black},
  tabsize=2,
  captionpos=b,
  breaklines=true,
  breakatwhitespace=false,
  title=\lstname,
  keywordstyle=\color{blue},
  commentstyle=\color{green!40!black},
  stringstyle=\color{red},
  keepspaces=true,
  xleftmargin=1.0cm,
  xrightmargin=1.0cm,
  framexleftmargin=0.5cm,
  framexrightmargin=0.5cm,
  literate={а}{а}1 {б}{б}1 {в}{в}1 {г}{г}1 {д}{д}1 {е}{е}1 {ё}{ё}1
           {ж}{ж}1 {з}{з}1 {и}{и}1 {й}{й}1 {к}{к}1 {л}{л}1 {м}{м}1 {н}{н}1
           {о}{о}1 {п}{п}1 {р}{р}1 {с}{с}1 {т}{т}1 {у}{у}1 {ф}{ф}1 {х}{х}1
           {ц}{ц}1 {ч}{ч}1 {ш}{ш}1 {щ}{щ}1 {ъ}{ъ}1 {ы}{ы}1 {ь}{ь}1 {э}{э}1
           {ю}{ю}1 {я}{я}1
           {А}{А}1 {Б}{Б}1 {В}{В}1 {Г}{Г}1 {Д}{Д}1 {Е}{Е}1 {Ё}{Ё}1
           {Ж}{Ж}1 {З}{З}1 {И}{И}1 {Й}{Й}1 {К}{К}1 {Л}{Л}1 {М}{М}1 {Н}{Н}1
           {О}{О}1 {П}{П}1 {Р}{Р}1 {С}{С}1 {Т}{Т}1 {У}{У}1 {Ф}{Ф}1 {Х}{Х}1
           {Ц}{Ц}1 {Ч}{Ч}1 {Ш}{Ш}1 {Щ}{Щ}1 {Ъ}{Ъ}1 {Ы}{Ы}1 {Ь}{Ь}1 {Э}{Э}1
           {Ю}{Ю}1 {Я}{Я}1
}

\usepackage{xcolor} % для управления цветами
\usepackage{tcolorbox} % для создания рамок
\usepackage{tikz} % Подключение пакета tikz
\usetikzlibrary{positioning} % Для позиционирования узлов (например, below left=of ...)

% Подключение пакета float для использования опции H
\usepackage{float}

%%% Для работы с русскими текстами (расстановки переносов, последовательность комманд строго обязательна)
\usepackage{microtype} % Улучшенное выравнивание и переносы
%\usepackage{polyglossia}
%\setdefaultlanguage{russian}
%\newfontfamily{\cyrillicfontt}{GOST_B}
%\set{GOST_type_A}

% 1. Создаем логическую переменную
\newif\ifGOSTfont
\GOSTfontfalse % по умолчанию выключено

% 2. Макрос для переключения шрифтов
\newcommand{\SwitchFonts}{%
  \ifGOSTfont
    % Если GOSTfont=true, ставим шрифты ГОСТ
    \setmainfont{GOST_type_A}
    \setromanfont{GOST_type_A}
    \setsansfont{GOST_type_A}
    \setmonofont{GOST_type_A}
  \else
    % Иначе стандартный Times New Roman
    \setmainfont{Times New Roman}
    \setromanfont{Times New Roman}
    \setsansfont{Times New Roman}
    \setmonofont{Times New Roman}
  \fi
}

% 3. Макросы для включения и выключения GOST шрифтов
\newcommand{\UseGOSTFont}{\GOSTfonttrue\SwitchFonts}
\newcommand{\UseTimesFont}{\GOSTfontfalse\SwitchFonts}

%\usepackage[numbercenter]{eskdplain} %отключить рамки

% polyglossia only
% \newfontfamily\cyrillicfont{GOST_type_A} 
% \newfontfamily\cyrillicfontrm{GOST_type_A}
% \newfontfamily\cyrillicfonttt{GOST_type_A}
% \newfontfamily\cyrillicfontsf{GOST_type_A}
%\defaultfontfeatures{Mapping=tex-text}

%%% Для работы со сложными формулами
\usepackage{amsmath}
\numberwithin{equation}{section} % Нумерация формул вида (раздел.номер)
\usepackage{amssymb}

%%% Глобально отключаем курсив и жирный для текста и цифр в формулах
% Используем пакет unicode-math для управления стилями
\usepackage{unicode-math}
% Устанавливаем основной математический шрифт и стиль
\setmathfont{Times New Roman}
\unimathsetup{math-style=TeX, bold-style=TeX}
% Отключаем курсив для букв и цифр в формулах
\everymath{\displaystyle\mathrm{}}
\everydisplay{\displaystyle\mathrm{}}

%%% Что бы использовать символ градуса (°) - \degree
\usepackage{gensymb}


%%% Для переноса составных слов
%\XeTeXinterchartokenstate=1
\XeTeXcharclass `\- 24
\XeTeXinterchartoks 24 0 ={\hskip\z@skip}
\XeTeXinterchartoks 0 24 ={\nobreak}

% Глобально запрещаем перенос слов с дефисом
\hyphenpenalty=10000
\exhyphenpenalty=10000

% Для листингов разрешаем переносы по дефису с помощью etoolbox
% \ПередНачаломОкружения{lstlisting}{% % etoolbox hook in Russian, if babel/polyglossia aliases are active
% \ПослеОкончанияОкружения{lstlisting}{\конецгруппы} % etoolbox hook in Russian
\BeforeBeginEnvironment{lstlisting}{%
  \begingroup
  \hyphenpenalty=50 % Стандартное значение LaTeX по умолчанию
  \exhyphenpenalty=50 % Стандартное значение LaTeX по умолчанию
}
\AfterEndEnvironment{lstlisting}{\endgroup}

%%% Ставим подпись к рисункам. Вместо «рис. 1» будет «Рисунок 1»
\addto{\captionsrussian}{\renewcommand{\figurename}{Рисунок}}
%%% Переименовываем заголовок библиографии
\addto{\captionsrussian}{\renewcommand{\refname}{Список используемых источников}}
%%% Убираем точки после цифр в заголовках
\def\russian@capsformat{%
  \def\postchapter{\@aftersepkern}%
  \def\postsection{\@aftersepkern}%
  \def\postsubsection{\@aftersepkern}%
  \def\postsubsubsection{\@aftersepkern}%
  \def\postparagraph{\@aftersepkern}%
  \def\postsubparagraph{\@aftersepkern}%
}

% Автоматически переносить на след. строку слова которые не убираются
% в строке
\sloppy

%%% Для вставки рисунков
\usepackage{graphicx}

%%% Для вставки интернет ссылок, полезно в библиографии
\usepackage{url}

% Packages for breakable tables using tabularx (via ltxtable)
\usepackage{array} % For advanced column specifications (e.g., >{\centering\arraybackslash}p{width})
\usepackage{tabularx} % For tables with adjustable-width X columns and p{width} columns
\usepackage{longtable} % For tables that span multiple pages
\usepackage{ltxtable} % Combines longtable and tabularx: allows tabularx environments to break across pages.
                      % Usage typically involves:
                      % \begin{filecontents*}{mytable.tex}
                      %   \begin{tabularx}{\linewidth}{...}
                      %     \caption{My Table Caption}\label{tab:my_table}\\ % Caption and label go inside
                      %     \hline
                      %     Col1 & Col2 \\ % Header row
                      %     \hline
                      %   \endfirsthead % Header for the first page
                      %
                      %   \multicolumn{2}{c}{Continuation of Table \ref{tab:my_table}} \\ % Example continuation caption
                      %   \hline
                      %   Col1 & Col2 \\ % Header row for subsequent pages
                      %   \hline
                      %   \endhead
                      %
                      %   \hline
                      %   \multicolumn{2}{r@{}}{Continued on next page} \\ % Footer for all pages except the last
                      %   \endfoot
                      %
                      %   \hline % Footer for the last page
                      %   \endlastfoot
                      %
                      %   % Table body
                      %   data & data \\
                      %   ...
                      %   \end{tabularx}
                      % \end{filecontents*}
                      % \LTXtable{\linewidth}{mytable.tex} % Or \LTXtable{\textwidth}{mytable.tex}
\usepackage{filecontents} % Helper to define content for ltxtable in the main .tex file

%%% Подподразделы(\subsubsection) выводим в содержании
\setcounter{tocdepth}{3}

%%% Настройка размера и стиля заголовков
\makeatletter
% Настройка стиля section с переходом на новую страницу
\renewcommand\section{\newpage\@startsection{section}{1}{\z@}%
  {-3.5ex \@plus -1ex \@minus -.2ex}%
  {1.0ex \@plus.2ex}%
  {\normalfont\normalsize\bfseries\hspace{1.7cm}}}

% Настройка стиля subsection с нумерацией
\renewcommand\subsection{\@startsection{subsection}{2}{\z@}%
  {-3.25ex\@plus -1ex \@minus -.2ex}%
  {0.1ex}%
  {\normalfont\normalsize\bfseries\hspace{1.7cm}}%
}

% Настройка стиля subsubsection без лишнего вертикального отступа после заголовка
\renewcommand\subsubsection{\@startsection{subsubsection}{3}{\z@}%
  {-2.5ex\@plus -1ex \@minus -.2ex}%
  {0.1ex}%
  {\normalfont\normalsize\bfseries\hspace{1.7cm}}%
}

% Настройка стиля paragraph
\renewcommand\paragraph{\@startsection{paragraph}{4}{\z@}%
  {0pt}%                                 Space before: set to 0pt
  {0.1ex}%                               Space after (if on its own line)
  {\normalfont\normalsize\bfseries\hspace{1.7cm}}%
}
\makeatother

%%% В введении нумерация подразделов идёт с буквой «В» (например В.1)
\makeatletter
\renewcommand\thesubsection{\ifnum\c@section=0{В.\arabic{subsection}}\else{\arabic{section}.\arabic{subsection}}\fi}
\makeatother

% Подключение пакета для управления отступами заголовков
\usepackage{titlesec}
% Сдвиг section и subsection на 1.7cm (номер и заголовок)
\titlespacing*{\section}{1.7cm}{*0}{*0}
\titleformat{\section}[hang]{\normalfont\normalsize\bfseries}{\thesection}{0.7em}{}
\titlespacing*{\subsection}{1.7cm}{*0}{*0}
\titleformat{\subsection}[hang]{\normalfont\normalsize\bfseries}{\thesubsection}{0.7em}{}


%%% Глобальная настройка списков itemize для отступа первой строки и переноса от левого края
%\NeedsTeXFormat{LaTeX2e}
\ProvidesPackage{itemize-setup}[2025/06/08 Custom itemize setup using enumitem for eskdx style]

\RequirePackage{enumitem}

% Ensure this package is loaded AFTER eskdx.tex so \leftmargini etc. and \parindent are defined.
% Default eskdx values (for reference from eskdtext.sty):
% \parindent is often 1.25cm (set by eskdx options)
% \leftmargini 2.5em
% \leftmarginii 2.2em
% \leftmarginiii 1.87em
% \leftmarginiv 1.7em
% \labelsep .5em (for all list environments)

% Vertical spacing parameters from eskdx's \@listI (for first level)
% \topsep 5\p@ \@plus2\p@ \@minus4\p@
% \parsep 2\p@ \@plus\p@ \@minus\p@
% \itemsep \parsep

% General settings for all levels of itemize
\setlist[itemize]{%
  align=parleft,        % Subsequent lines of item body align with first line of item body.
  listparindent=0pt,    % No paragraph indentation within the item body.
  labelsep=0.5em,       % eskdx default for labelsep.
  % Vertical spacing (simplified from eskdx defaults):
  topsep = 5pt, 
  parsep = 2pt, 
  itemsep = 2pt, % eskdx sets itemsep = parsep
  nolistsep=false       % eskdx doesn't seem to use an equivalent of nolistsep.
}

% Level 1: em-dash. Indent matches document's paragraph indent.
\setlist[itemize,1]{%
  label=\textemdash,
  leftmargin=\parindent   % Use document's paragraph indent for the first level list margin.
}

% Level 2: en-dash. Uses eskdx's specific indent for level 2.
\setlist[itemize,2]{%
  label=\textendash,
  leftmargin=\leftmarginii % This is an absolute indent from the main text margin.
}

% Level 3: asterisk. Uses eskdx's specific indent for level 3.
\setlist[itemize,3]{%
  label=*,
  leftmargin=\leftmarginiii % Absolute indent.
}

% Level 4: period. Uses eskdx's specific indent for level 4.
\setlist[itemize,4]{%
  label=.,
  leftmargin=\leftmarginiv  % Absolute indent.
}

% Sanity check for eskdx lengths and \parindent at begin document
% This helps confirm that eskdx.sty (which defines these) is loaded before this file,
% and that \parindent has a non-zero value.
\AtBeginDocument{%
  \ifdim\parindent=0pt
    \PackageWarning{itemize-setup}{Document \string\parindent\space is 0pt. First level itemize indent will also be 0pt. Ensure \string\parindent\space is set correctly for GOST (e.g., 1.25cm).}
  \fi
  \@ifundefined{leftmarginii}{\PackageWarning{itemize-setup}{\string\leftmarginii\space is not defined. eskdx might not be loaded before itemize-setup.}}{}%
  \@ifundefined{leftmarginiii}{\PackageWarning{itemize-setup}{\string\leftmarginiii\space is not defined.}}{}%
  \@ifundefined{leftmarginiv}{\PackageWarning{itemize-setup}{\string\leftmarginiv\space is not defined.}}{}%
}

\endinput
%%
%% End of file `itemize-setup.tex'.

%
% Настройка отступов для itemize с использованием enumitem
% \leftskip и \rightskip уже установлены глобально
\setlist[itemize]{
  labelsep=0.5em,     % Стандартный отступ метки
  rightmargin=0.5cm,    % Правый край списка совпадает с \rightskip
  topsep=0pt, partopsep=0pt, parsep=0pt, itemsep=0pt % Убираем пустые строки и делаем межстрочный интервал 1.0
}
\setlist[itemize,1]{leftmargin=0.5cm, itemindent=1.5cm}
\setlist[itemize,2]{leftmargin=0.5cm, itemindent=2.0cm}
\setlist[itemize,3]{leftmargin=0.5cm, itemindent=4.5cm}
\setlist[itemize,4]{leftmargin=0.5cm, itemindent=6.0cm}
% Альтернативная настройка enumerate без enumitem
% \makeatletter
% \renewcommand{\theenumi}{\asbuk{enumi}}
% \renewcommand{\labelenumi}{\theenumi)}
% \makeatother

% Настройка отступов для description с использованием enumitem (аналогично itemize)
\setlist[description]{
  labelsep=0.5em,
  rightmargin=0.5cm,
  topsep=0pt, partopsep=0pt, parsep=0pt, itemsep=0pt
}
\setlist[description,1]{leftmargin=0.5cm, itemindent=1.5cm}
\setlist[description,2]{leftmargin=0.5cm, itemindent=2.0cm}
\setlist[description,3]{leftmargin=0.5cm, itemindent=4.5cm}
\setlist[description,4]{leftmargin=0.5cm, itemindent=6.0cm}




\newcommand{\ESKDNNTCcitydate}[2]{%
  \def\@ESKDNNTCcity{#1}
  \ESKDdate{#2}
}

\renewcommand{\ESKDtheTitleFieldX}{\@ESKDNNTCcity, \ESKDtheYear}

\newcommand{\ESKDNNTCtitle}[3]{%
  \renewcommand{\ESKDtheTitleFieldVIII}{%
    \fontsize{14pt}{16pt}\selectfont
    \noindent
    \begin{minipage}[t]{0.45\textwidth}
      Выполнил студент группы\\
      #1\\
      #2\\[2mm]
    \end{minipage}%
    \hfill
    \begin{minipage}[t]{0.45\textwidth}
      Проверил преподаватель\\
      #3\\
      Проект защищен с оценкой: \makebox[6mm]{\hrulefill}\\
      Дата защиты: \underline{\hspace{30mm}}\\
    \end{minipage}%

    \vspace{5mm}

    \noindent
    \begin{minipage}[t]{0.45\textwidth}
      \mbox{Подпись студента:} \underline{\hspace{20mm}}
    \end{minipage}%
    \hfill
    \begin{minipage}[t]{0.45\textwidth}
      \mbox{Подпись преподавателя:} \underline{\hspace{20mm}}
    \end{minipage}%
  }
}

\makeatletter
\newcommand{\ESKDNNTCspeciality}[1]{%
  \def\ESKDNNTC@speciality{#1}%
}
\newcommand{\ESKDNNTCqualification}[1]{%
  \def\ESKDNNTC@qualification{#1}%
}
\newcommand{\ESKDNNTCsubject}[1]{%
  \def\ESKDNNTC@subject{#1}%
}
\newcommand{\ESKDNNTCwork}[1]{
  \def\ESKDNNTC@work{#1}
}

\renewcommand{\ESKDtheTitleFieldIV}{
  \ESKDNNTC@work\par
  \vspace{2mm}   
  \ESKDNNTC@subject\par
  \vspace{2mm}
  \ESKDtheDocName
  \vspace{170mm}
}

\renewcommand{\ESKDtheTitleFieldI}{%
  \ESKDtheDepartment\par
  \ESKDtheCompany\par
  \vspace{5mm}
  \ESKDNNTC@speciality\par
  \ESKDNNTC@qualification
}
\makeatother


% Добавить надписи снизу
% Макрос для колонки XXXI
\newcommand{\ESKDNNTCsetFirstFooter}[1]{%
  \renewcommand{\ESKDcolumnXXXIname}{#1}%
}

% Макрос для колонки XXXII
\newcommand{\ESKDNNTCsetSecondFooter}[1]{%
  \renewcommand{\ESKDtheColumnXXXII}{#1}%
}






