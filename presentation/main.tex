\documentclass{beamer}

\usepackage{xcolor}
\usepackage{amsmath}
\usepackage{fontspec}

% Define custom colors
\definecolor{spaceblue}{rgb}{0.1, 0.1, 0.3}   % Dark blue for space background
\definecolor{starwhite}{rgb}{1.0, 1.0, 1.0}   % White for stars and text
\definecolor{planetgreen}{rgb}{0.2, 0.6, 0.2} % Green for highlights

% Set up the Beamer theme
\setbeamertemplate{background}{
  \begin{tikzpicture}[overlay, remember picture]
    \fill[spaceblue] (current page.south west) rectangle (current page.north east); % Dark background
  \end{tikzpicture}
}

\setbeamercolor{item}{fg=starwhite}  % Color for itemize
\setbeamercolor{block title}{fg=starwhite, bg=spaceblue!80} % Block titles
\setbeamercolor{block body}{fg=starwhite, bg=spaceblue!60}  % Block body
\setbeamercolor{structure}{fg=planetgreen} % Highlight color for sections
\setbeamercolor{section in toc}{fg=planetgreen}  % Section colors in the table of contents

% Set the font to something more futuristic (if using XeLaTeX or LuaLaTeX)
\setmainfont{Fira Sans}  % You can replace this with another font of your choice

% Title page customizations
\title{Exploration of the Universe}
\author{Ivan Ivanov}
\date{\today}

\begin{document}

% Title page
\begin{frame}
  \titlepage
\end{frame}

% Slide 1: What is Space?
\begin{frame}
  \frametitle{What is Space?}
  Space is the vast, seemingly infinite expanse that surrounds Earth and contains all stars, planets, galaxies, and other celestial bodies.
  
  \begin{itemize}
    \item \textbf{Stars}: Massive hot balls of gas.
    \item \textbf{Planets}: Bodies that orbit stars.
    \item \textbf{Black holes}: Regions of space with incredibly strong gravitational pull.
  \end{itemize}
\end{frame}

% Slide 2: Why Explore Space?
\begin{frame}
  \frametitle{Why Explore Space?}
  Space exploration allows us to:
  \begin{itemize}
    \item \alert{Understand the origins of the Universe.}
    \item \alert{Discover new technologies that can benefit life on Earth.}
    \item \alert{Access potential resources from other planets and asteroids.}
  \end{itemize}
\end{frame}

% Slide 3: The Future of Space Exploration
\begin{frame}
  \frametitle{The Future of Space Exploration}
  In the near future, we can expect:
  \begin{enumerate}
    \item Spacecraft capable of traveling to Mars.
    \item Advances in technology for studying exoplanets.
    \item Missions aimed at mining asteroids for valuable resources.
  \end{enumerate}
\end{frame}

% Slide 4: Key Milestones
\begin{frame}
  \frametitle{Key Milestones in Space Exploration}
  The major milestones that have shaped modern space exploration:
  \begin{itemize}
    \item \textbf{1957}: Launch of Sputnik, the first artificial satellite.
    \item \textbf{1969}: The Moon landing by Apollo 11.
    \item \textbf{2020}: Mars Rover Perseverance lands on Mars.
  \end{itemize}
\end{frame}

% Slide 5: Conclusion
\begin{frame}
  \frametitle{Conclusion}
  Space exploration is one of humanity's most exciting endeavors, expanding our knowledge of the universe and pushing the boundaries of technology. The future holds immense potential for discovery and innovation.
\end{frame}

\end{document}
